\documentclass[a4paper,12pt]{article}
\usepackage{amsmath, amssymb, geometry}
\geometry{margin=1in}

\title{Problem Set 2}
\author{Juan F. Imbet Ph.D.}
\date{}

\begin{document}
	
	\maketitle
	
	\begin{center}
		\textbf{Empirical Asset Pricing} \\
		\textbf{M2 104} \\
		\textbf{Paris Dauphine - PSL} \\
	\end{center}
	
	\noindent The problem set together with the code needs to be emailed to \texttt{juan.imbet@dauphine.psl.eu} \textbf{before March 7, 23:59}. You can solve the problem sets in groups of maximum \textit{2} people. If you cannot email it, use a we transfer link.
	
	\section*{Setup}
	
	Consider the following factor model with 5 assets and 2 factors with the appropriate dimensions of the parameters:
	\begin{align*}
		\underbrace{R_{t}^e}_{5 \times 1} &= \underbrace{a}_{5 \times 1} + \underbrace{\beta}_{5 \times 2} \underbrace{f_t}_{2 \times 1} + \underbrace{\epsilon_{t}}_{5 \times 1} \\
	\end{align*}
	with $\epsilon_t \sim \mathcal{N}(0, \Sigma)$, where $\Sigma$ is a non-diagonal covariance matrix. And $f_t \sim \mathcal{N}(\mu_f, \Sigma_f)$, where $\Sigma_f$ is the covariance matrix of the factor realizations, and $\mu_f$ is the expected value of the factor returns.
	
	The true values of the parameters are:
	\begin{align*}
		a &= \begin{pmatrix} 0.0 \\ 0.0 \\ 0.0 \\ 0.0 \\ 0.0 \end{pmatrix}, \quad 
		\beta = \begin{pmatrix} 0.5 & 0.0 \\ 0.0 & 0.5 \\ 0.5 & 0.5 \\ 0.3 & 1.2 \\ 0.7 & 0.4 \end{pmatrix}, \\
		\Sigma &= \begin{pmatrix} 1.0 & 0.5 & 0.5 & 0.5 & 0.5 \\ 0.5 & 1.0 & 0.5 & 0.0 & 0.0 \\ 0.5 & 0.5 & 1.0 & 0.0 & 0.0 \\ 0.5 & 0.0 & 0.0 & 1.0 & 0.5 \\ 0.5 & 0.0 & 0.0 & 0.5 & 1.0 \end{pmatrix}
	\end{align*}
	and
	\begin{align*}
		\mu_f &= \begin{pmatrix} 0.05 \\ 0.07 \end{pmatrix}, \quad 
		\Sigma_f = \begin{pmatrix} 1.0 & 0.5 \\ 0.5 & 1.0 \end{pmatrix}
	\end{align*}
	
	\section*{Question 1 (4 points)}
	Create a function that given a time horizon $T$ simulates the dynamics of the system above. The function should return both time series $R_t^e$ and $f_t$.
	
	\newpage
	\section*{Question 2 (4 points)}
	Create a function that given simulated data $R_t^e$ estimates using OLS and GLS the parameters $\hat{\alpha}$ and $\hat{\lambda}$ (together with their standard errors) in the following model (assume that you know the true values of $\Sigma$, $\Sigma_f$ and $\beta$ so you don't need to estimate them):
	\begin{equation*}
		E_T[R_{t}^e] = \alpha + \beta \lambda
	\end{equation*}
	
	\section*{Question 3 (4 points)}
	For a given $T=10$ repeat the above exercise 1000 times and plot the distribution of the estimated parameters $\hat{\alpha}$ and $\hat{\lambda}$ (together with the true values). What do you observe? The true values of $\alpha$ are $a$ and the true values of $\lambda$ are $\mu_f$.
	
	\section*{Question 4 (4 points)}
	Repeat the above exercise, but for each estimated parameter consider the distribution of the ratio
	\begin{equation*}
		\frac{\hat{\theta}-\theta}{\text{s.e.}(\hat{\theta})}
	\end{equation*}
	Can you find any difference in the distribution of the ratios between OLS and GLS? Hint: look at the tails of the distribution. Comment on the results.
	
	\section*{Question 5 (4 points)}
	Assume now that you do not know the true values of $\beta$, $\Sigma$ and $\Sigma_f$. For a fixed $T=10$ and 1000 simulations, compare the expected value of your estimators. Does estimation error affect the expected value of $\hat{\alpha}$ and $\hat{\lambda}$? Comment on the results.
	
\end{document}
